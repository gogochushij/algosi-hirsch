\documentclass[a4paper]{article}

\usepackage[english, russian]{babel}
\usepackage[utf8]{inputenc}
\usepackage{amsmath,amssymb,amsthm,mathtools}
\usepackage{graphicx,caption}
\usepackage[ruled,linesnumbered,vlined]{algorithm2e}
\usepackage{comment}
\usepackage{hyperref}
\usepackage{epigraph}
\usepackage{enumitem}


% Margins
\usepackage[top=2.5cm, left=3cm, right=3cm, bottom=3.0cm]{geometry}
% Colour table cells
\usepackage[table]{xcolor}
\definecolor{cadmiumgreen}{rgb}{0.0, 0.42, 0.24}
\definecolor{pigmentblue}{rgb}{0.2, 0.2, 0.6}
\hypersetup{ %use colored text instead of ugly boxes
	colorlinks,
	linkcolor={cadmiumgreen},
	urlcolor={pigmentblue}
}

\usepackage{tikz}
\usepackage{pdfpages}

% My commands
\newcommand{\e}{\varepsilon}
\newcommand{\heart}{\ensuremath\heartsuit}
\newcommand{\secauthor}[1]{
\begin{center}
    \footnotesize{\textit{(Автор(ы): #1)}}
\end{center}
}
%%%Author tags
\newcommand{\denis}{Осипов Денис}
\newcommand{\egor}{Нечаев Егор}
\newcommand{\ivan}{Ермошин Иван}
%%%

% My commands (continued)
\newtheorem{theorem}{Теорема}[section]
\newtheorem{definition}{Определение}[section]
\newtheorem{example}{Пример}[section]
\newtheorem{problem}{Задача}[section]
\newtheorem{lemma}{Лемма}[section]
\newtheorem{corollary}{Следствие}[section]
\newtheorem{nb}{NB}[section]

\newtheorem*{theorem*}{Теорема}
\newtheorem*{definition*}{Определение}
\newtheorem*{example*}{Пример}
\newtheorem*{problem*}{Задача}
\newtheorem*{lemma*}{Лемма}
\newtheorem*{nb*}{NB}

% My environments
\newcommand*{\QEDTRIANGLE}{\null\nobreak\hfill\ensuremath{\bigtriangleup}}
\newenvironment{algodescription}[1]
    {\noindent\textbf{#1.}}
    {\QEDTRIANGLE}



%%%%%%%%%%%%%%%%%
%     Title     %
%%%%%%%%%%%%%%%%%
\title{Алгосы, часть ii}
\author{Денис Осипов, Иван Ермошин, Егор Нечаев, Эдуард Алексеевич Гирш}
\date{\today}

\begin{document}

\includepdf{figures/frontcover.pdf}

\maketitle

\section*{Введение}

Этот проект -- коллективный \textbf{конспект} по второй части курса <<Математические основы алгоритмов>>, впервые прочитанного первокурсникам МКН СПбГУ в первой половине ii семестра 2020 года Эдуардом Алексеевичем Гиршем. 

Актуальные исходники: \url{https://www.overleaf.com/read/hnbkrkyknbpk} и \url{https://github.com/gogochushij/algosi-hirsch}

Если вы хотите \textbf{принять участие} в написании билетов, или же 
\textbf{сообщить об ошибке}, напишите \url{http://vk.com/gogochushij}. Предполагается, что каждый автор напишет около 4 билетов, но мы будем рады любой посильной помощи. \href{https://docs.google.com/spreadsheets/d/17MKhLVzCyYvEKlm6W5Bb-6uUDNyv0QLmdvuh4N6JfXI/edit?usp=sharing}{\texttt{Здесь}} можно посмотреть, с какими билетами вы можете помочь проекту.

Последнее обновление публичной версии конспекта: \date{\today}

\tableofcontents \newpage

%%%%%%%%%%%%%%%%%%%%%%%%%%%%%%%%%%%%%%%%%
% Здесь настроить конспект для компиляции
\foreach \n in {1,...,18,20,21} {\input{que\n.tex}}
%%%%%%%%%%%%%%%%%%%%%%%%%%%%%%%%%%%%%%%%%

\section*{Вопросы к ii части экзамена}
\addcontentsline{toc}{section}{Вопросы к ii части экзамена}
\begin{enumerate}[wide, labelwidth=!, labelindent=0pt]

\item Булевы схемы как модель параллельных вычислений. Принцип Брента. Параллельное умножение булевых матриц. Параллельный алгоритм для задачи достижимости в графе.

\item Параллельное вычисление всех префиксов суммы. Параллельное сложение и умножение чисел.

\item Параллельное вычисление номера элемента списка с конца. Параллельное вычисление глубин всех вершин дерева.

\item Приближенный алгоритм для задачи о рюкзаке.

\item Приближенные алгоритмы для задачи о покрытии множествами (Set Cover): сведение к задаче линейного программирования (со следствием для задачи вершинного покрытия (Vertex Cover)), двойственная задача, прямо-двойственный (primal-dual) метод.

\item Жадный приближенный алгоритм для задачи о вершинном покрытии.

\item Задача о максимальном потоке. Теорема о максимальном потоке и минимальном разрезе. Алгоритм Форда-Фалкерсона. Случай целочисленных весов. Применение к задаче о максимальном паросочетании в двудольном графе.

\item Алгоритм Эдмондса-Карпа для задачи о максимальном потоке.

\item Алгоритм проталкивания предпотока.

\item Приближённые алгоритмы для задачи коммивояжера в метрическом пространстве (без алгоритмов для минимального остовного дерева и минимального совершенного паросочетания).

\item Алгоритмы Прима и Крускала для задачи о минимальном остовном дереве.

\item Вероятностные алгоритмы с односторонней ограниченной вероятностью ошибки. Алгоритм Фрейвальдса для проверки умножения матриц.

\item Вероятностный алгоритм для сравнения строк на расстоянии и алгоритм Рабина-Карпа.

\item Вероятностный вариант быстрой сортировки (Randomized QuickSort).

\item Проверка равенства полиномов. Лемма Шварца-Циппеля.

\item Вероятностная проверка простоты числа: алгоритм Соловея-Штрассена.

\item Хеш-таблицы. Универсальные семейства хеш-функций и их применение к хеш-таблицам.

\item Совершенное хеширование.

\item (нет в этом конспекте) Алгоритм Борувки и линейный вероятностный алгоритм для минимального остовного дерева (без алгоритма верификации и поиска тяжелых рёбер).

\item Слабоэкспоненциальный детерминированный алгоритм для 3-SAT.

\item Слабоэкспоненциальный вероятностный алгоритм для 3-SAT (алгоритм Шонинга, использующий случайное блуждание).

\item (нет в этом конспекте) Онлайн-алгоритм для задачи кеширования.

\item (нет в этом конспекте) Алгоритм для онлайн-варианта задачи о покрытии множествами
    
\end{enumerate}


\section*{Источники и программа ii части экзамена}
\addcontentsline{toc}{section}{Источники и программа ii части экзамена}

\newcommand{\smallblue}[1]{\noindent {\footnotesize\color{blue} #1}}

\subsection*{Параллельные алгоритмы}
\begin{enumerate}[wide, labelwidth=!, labelindent=0pt]

\item Булевы схемы как модель параллельных вычислений. Принцип Брента. Умножение булевых матриц. Достижимость в графе (через возведение матрицы в степень). Вычисление всех префиксов суммы. Параллельное сложение и умножение чисел. 

\smallblue{Литература: Christos Papadimitriou. Computational Complexity. Addison-Wesley, 1994. Section 15.1.
}

\item Вычисление номера элемента списка с конца. Вычисление глубин всех вершин дерева.

\smallblue{Литература: Thomas H. Cormen, Charles E. Leiserson, and Ronald L. Rivest. Introduction to Algorithms. MIT Press, 1990. Section 30.1. \url{http://staff.ustc.edu.cn/~csli/graduate/algorithms/book6/chap30.htm}

Есть русский перевод, но нужно именно первое издание – в последующих параллельных вычислений нет: \url{https://e-maxx.ru/bookz/files/cormen.pdf}
}
\end{enumerate}

\subsection*{Оптимизационные задачи и приближенные алгоритмы}

\begin{enumerate}[wide, labelwidth=!, labelindent=0pt] \setcounter{enumi}{2}

\item Приближенные алгоритмы для задачи о рюкзаке (Knapsack).

\smallblue{Литература: С. Дасгупта, Х. Пападимитриу, У. Вазирани. Алгоритмы. (В пер. А.С.Куликова). МЦНМО, 2014. \url{https://alexanderskulikov.github.io/files/algorithms_href.pdf}. Раздел 9.2.4.

Альтернатива --- конспект \url{https://logic.pdmi.ras.ru/~hirsch/students/effalg-2001/lecture9.pdf}.Раздел 9.1.
}

\item Приближенные алгоритмы для задачи о покрытии множествами (Set Cover): сведение к задаче линейного программирования (со следствие для задачи вершинного покрытия (Vertex Cover)), двойственная задача, прямо-двойственный (primal-dual) метод, жадный алгоритм.

\smallblue{Литература: David P. Williamson and David B. Shmoys. The Design of Approximation Algorithms. Cambridge University Press, 2010. Sections 1.2-1.6.

Жадный алгоритм есть также в конспекте \url{https://logic.pdmi.ras.ru/~hirsch/students/effalg-2001/lecture9.pdf}. Раздел 9.2.
}

\item Задача о максимальном потоке. Теорема о максимальном потоке и минимальном разрезе. Алгоритм Форда-Фалкерсона. Случай целочисленных весов. Применение к задаче о максимальном паросочетании в двудольном графе. Алгоритм Эдмондса-Карпа. Алгоритм проталкивания предпотока.

\smallblue{Литература: Thomas H. Cormen, Charles E. Leiserson, and Ronald L. Rivest. Introduction to Algorithms. MIT Press, 1990. Chapter 27. Есть русский перевод. Эти алгоритмы есть в любом издании, в зависимости от издания глава 26 или другая.

Альтернатива --- John Kleinberg, Eva Tardos. Algorithm Design. Pearson, 2006. Sections 7.1-7.5. \url{http://www.cs.sjtu.edu.cn/~jiangli/teaching/CS222/files/materials/Algorithm\%20Design.pdf}
}

\item Приближённые алгоритмы для задачи коммивояжера в метрическом пространстве (без алгоритмов для минимального остовного дерева и минимального совершенного паросочетания).

\smallblue{Литература: конспект \url{https://logic.pdmi.ras.ru/~hirsch/students/effalg-2001/lecture9.pdf}. Раздел 9.4.
}

\item Алгоритмы Прима и Крускала для задачи о минимальном остовном дереве.

\smallblue{Литература: Thomas H. Cormen, Charles E. Leiserson, and Ronald L. Rivest. Introduction to Algorithms. MIT Press, 1990. Chapter 24.

Есть русский перевод. Эти алгоритмы есть в любом издании, в зависимости от издания глава 23 или другая.
}
\end{enumerate}

\subsection*{Вероятностные алгоритмы}

\begin{enumerate}[wide, labelwidth=!, labelindent=0pt] \setcounter{enumi}{7}

\item Вероятностные алгоритмы с односторонней ограниченной вероятностью ошибки. Алгоритм Фрейвальдса для проверки умножения матриц.

\smallblue{Литература: конспект \url{https://logic.pdmi.ras.ru/~hirsch/students/effalg-2001/lecture1.pdf}. Раздел 1.3.
}

\item Сравнение строк на расстоянии и алгоритм Рабина-Карпа.

\smallblue{Литература: конспект \url{https://logic.pdmi.ras.ru/~hirsch/students/effalg-2001/lecture1.pdf}. Раздел 1.5.
}

\item Randomized QuickSort.

\smallblue{Литература: Thomas H. Cormen, Charles E. Leiserson, and Ronald L. Rivest. Introduction to Algorithms. MIT Press, 1990. Chapter 8. 

Есть русский перевод. Этот алгоритм есть в любом издании, в зависимости от издания глава 7 или другая.
}

\item Проверка равенства полиномов. Лемма Шварца-Циппеля.

\smallblue{Литература: М.Солтис. Введение в анализ алгоритмов. ДМК, 2019. Лемма 6.3.
}

\item Вероятностная проверка простоты: алгоритм Соловея-Штрассена.

\smallblue{Литература: конспект \url{https://logic.pdmi.ras.ru/~hirsch/students/effalg-2001/lecture8.pdf} Раздел 8.1.

(там отсутствует доказательство критерия Эйлера, но его можно посмотреть хоть в википедии \url{https://en.wikipedia.org/wiki/Euler\%27s_criterion})
}

\item Хеш-таблицы. Универсальные семейства хеш-функций. Совершенное хеширование.

\smallblue{Литература: М.А.Бабенко, М.В.Левин: Введение в теорию алгоритмов и структур данных. МЦНМО, 2016. Раздел 5. Альтернатива – всё, кроме совершенного хеширования, есть в Thomas H. Cormen, Charles E. Leiserson, and Ronald L. Rivest. Introduction to Algorithms. MIT Press, 1990. Разделы 12.2, 12.3.3.

Есть русский перевод. Это есть в любом издании, в зависимости от издания глава 11 или другая.
}

\item Линейный вероятностный алгоритм для минимального остовного дерева (без алгоритма верификации).

\smallblue{Литература: конспект \url{https://logic.pdmi.ras.ru/~hirsch/students/effalg-2001/lecture6.pdf}. Раздел 6.2
}

\item Слабоэкспоненциальные алгоритмы для 3-SAT (детерминированные и вероятностный).

\smallblue{
Литература: \href{https://drive.google.com/file/d/1VuiRy7RSX9AbnaDgrAnNO5tQFhuF38tQ/view?usp=sharing}{слайды}
}

\end{enumerate}

\subsection*{Online-алгоритмы}

\begin{enumerate}[wide, labelwidth=!, labelindent=0pt] \setcounter{enumi}{15}
\item Алгоритм для задачи кеширования.

\smallblue{Материалы: \href{https://drive.google.com/file/d/1zswiwOhxnbeAKITohZ2-8xOjcsLIjJhY/view?usp=sharing}{видео}, \href{https://drive.google.com/file/d/1-8KPQu9iJFFUzFjeEqdbPSt6bRBW10tT/view?usp=sharing}{слайды}
}

\item Алгоритм для онлайн-варианта задачи о покрытии множествами (случай единичных весов и общий случай).

\smallblue{Материалы: \href{https://drive.google.com/file/d/1Iyor1MbXwP4o6hvtSkZyB0Ql5K-bFLjN/view?usp=sharing}{видео}, \href{https://drive.google.com/file/d/1nuRkh5KggZCWTM-5ub2BSeEoF9O2ds6c/view?usp=sharing}{слайды}
}
\end{enumerate}

% que6.tex ссылается на que5.tex
% que21.tex ссылается на que20.tex
% que9.tex -> que8.tex -> que7.tex

\end{document}
