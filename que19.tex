\section{(19) (В РАЗРАБОТКЕ) Алгоритм Борувки для MST. Линейный вероятностный алгоритм для MST. (Осипов Д.)}

\newcommand{\achtung}{\color{red}\textbf{Warning!} Авторское доказательство}

\subsection{Алгоритм Борувки}
Шаг Борувки -- это алгоритм, который сводит задачу поиска миностова у графа к той же задаче, но с меньшим числом вершин у графа (иногда и с меньшим числом ребер). Алгоритм Борувки -- многократное применение шага Борувки. Шаг Борувки базируется на следующей лемме:

\begin{lemma*}[\hypertarget{baselemma}{о безопасных ребрах для алгоритма Борувки}] Для всякой вершины $v \in V$ хотя бы одно смежное с $v$ ребро минимального веса входит в любое минимальное остовное дерево. \end{lemma*}
\begin{proof}[\achtung]
Если все смежные в $v$ ребра имеют одинаковый вес, то доказывать нечего -- вершина $v$ должна быть покрыта хоть каким-то смежным с ней ребром. Пусть теперь среди смежных с $v$ ребер есть ребра веса, строго большего, чем минимальный.

Пусть $T$ -- какой-то минимальный остов. Предположим, что ни одно из ребер, смежных с $v$ и имеющих среди них минимальный вес, не входит в $T$. Пусть $(v, w)$ -- любое такое ребро. Так как $T$ -- остов, то вершина $v$ покрыта более тяжелым ребром из него, пусть $(v, u)$. За $P$ обозначим единственный путь из $u$ в $w$ по дереву $T$. Добавим $(v, w)$ в $T$, тогда $P + (w, v) + (v, u)$ есть цикл в $T\cup\{(v, w)\}$, проходящий через $v$. Удаление ребра $(u, v)$ разрушит этот цикл, и полученное множество ребер $T' = T\smallsetminus \{(v, u)\} \cup \{(v, w)\}$ будет снова остовным деревом. Но вес $T'$ будет строго меньше веса $T$ -- противоречие с минимальностью.
\end{proof}

\textbf{Шаг Борувки.}
\begin{enumerate}
    \item Для каждой вершины $v \in V$ помечаем смежное с ней ребро минимального веса. Если таких ребер несколько, выбираем ребро с наименьшим номером.
    \item Определим компоненты связности на помеченных ребрах.
    \item Каждую компоненту связности стянем в одну вершину. Некоторые ребра при этом станут петлями или мультиребрами.
    \item Все петли уберем, а в мультиребрах оставим только ребра минимального веса.
\end{enumerate}

Корректность алгоритма Борувки заключается в следующем утверждении:

\begin{theorem*} Пусть шаг Борувки получил из графа $G$ граф $G'$. Тогда миностов графа $G$ есть миностов графа $G'$ плюс помеченные в этом шаге Борувки ребра. \end{theorem*}
Для доказательства воспользуемся:
\begin{lemma*} Ребра, отмеченные на шаге Борувки, образуют лес. \end{lemma*}
\begin{proof}[\achtung ~леммы]

Предположим, что какие-то из отмеченных ребер образовали цикл. Ориентируем ребра этого цикла следующим образом. Пусть в шаге 1 для вершины $v$ было помечено ребро $(v, u)$, тогда ориентируем его как $v\rightarrow u$.

Утверждается, что получившийся орграф есть цикл в ориентированном смысле (а не, например, поток). Действительно, для каждой вершины $v$ в цикле верно $out(v) = 1$ по смыслу алгоритма и $in(v) + out(v) = 2$ по смыслу цикла, значит и $in(v) = 1$.

Итак, пусть имеем цикл $v_1 \rightarrow v_2 \rightarrow \ldots \rightarrow v_k \rightarrow v_1$. Ребро $(v_j, v_{j+1})$, ориентированное как $v_j \rightarrow v_{j+1}$, означает, что оно было выбрано как минимальное среди всех ребер, смежных с $v_j$, откуда имеем для весов $w(v_j, v_{j+1}) \leq w(v_j, v_{j-1})$. Применив это рассуждение для всех вершин в цикле, имеем: $$w(v_1, v_2) \geq w(v_2, v_3) \geq \ldots \geq w(v_{k-1}, v_k) \geq w(v_k, v_1) \geq w(v_1, v_2),$$
откуда следует, что у всех ребер цикла одинаковый вес.

Вспомним, что в случае нескольких смежных ребер с минимальным весом алгоритм выбирает ребро с наименьшим номером (см. шаг 1). Обозначим номер ребра через $\#$. Тогда имеем для всех~$j$ $\#(v_j, v_{j+1}) > \#(v_j, v_{j-1})$, или:
$$\#(v_1, v_2) > \#(v_2, v_3) > \ldots > \#(v_{k-1}, v_k) > \#(v_k, v_1) > \#(v_1, v_2),$$
откуда и получаем противоречие.
\end{proof}

\begin{proof}[\achtung ~теоремы]
Шаг Борувки в графе $G$ построил лес $F$, каждому дереву которого соответствует вершина в графе $G'$. Миностов $T'$ графа $G'$ соединяет все вершины графа $G'$, т.е. все деревья леса $F$ в $G$, поэтому объединение $F$ и $T'$ есть дерево. По \hyperlink{baselemma}{\textit{лемме о безопасных ребрах}} все ребра этого дерева входят в какой-то миностов $G$, ну значит этот миностов и есть $F \cup T'$.
\end{proof}

Теперь несложно получить оценку на время работы.

\begin{lemma*} Время работы шага Борувки есть $O(E+V)$. \end{lemma*}
\begin{proof}

Шаг 1 требует однократного просмотра всех смежных ребер у каждой вершины: $O(E+V)$.

Шаг 2 можно выполнить поиском в глубину, который работает за $O(E+V)$.

Шаг 3 требует переназначения вершин в новые компоненты связности -- $O(V)$ -- и перераспределения всех ребер на новые вершины  -- $O(E)$.

Шаг 4 требует просмотра всех ребер -- $O(E)$.
\end{proof}

В связном графе $V \leq E + 1 = O(E)$, так что верна и оценка $O(E)$.

Заметим, наконец, что всякий шаг Борувки уменьшает число вершин не менее, чем в два раза. Действительно, шаг 1 соединяет каждую вершину с какой-то, значим образуется не более $n/2$ компонент -- вершин в новом графе. Отсюда сразу следует, что применить шаг Борувки до построения полного миностова нужно не более $\log_2 V = O(\log V)$ раз. Общая оценка времени работы алгоритма Борувки есть $O(E\log V)$.

\subsection{Линейный вероятностный алгоритм для MST}

Coming soon
