\section*{Вопросы к ii части экзамена}
\addcontentsline{toc}{section}{Вопросы к ii части экзамена}
\begin{enumerate}[wide, labelwidth=!, labelindent=0pt]

\item Булевы схемы как модель параллельных вычислений. Принцип Брента. Параллельное умножение булевых матриц. Параллельный алгоритм для задачи достижимости в графе.

\item Параллельное вычисление всех префиксов суммы. Параллельное сложение и умножение чисел.

\item Параллельное вычисление номера элемента списка с конца. Параллельное вычисление глубин всех вершин дерева.

\item Приближенный алгоритм для задачи о рюкзаке.

\item Приближенные алгоритмы для задачи о покрытии множествами (Set Cover): сведение к задаче линейного программирования (со следствием для задачи вершинного покрытия (Vertex Cover)), двойственная задача, прямо-двойственный (primal-dual) метод.

\item Жадный приближенный алгоритм для задачи о вершинном покрытии.

\item Задача о максимальном потоке. Теорема о максимальном потоке и минимальном разрезе. Алгоритм Форда-Фалкерсона. Случай целочисленных весов. Применение к задаче о максимальном паросочетании в двудольном графе.

\item Алгоритм Эдмондса-Карпа для задачи о максимальном потоке.

\item Алгоритм проталкивания предпотока.

\item Приближённые алгоритмы для задачи коммивояжера в метрическом пространстве (без алгоритмов для минимального остовного дерева и минимального совершенного паросочетания).

\item Алгоритмы Прима и Крускала для задачи о минимальном остовном дереве.

\item Вероятностные алгоритмы с односторонней ограниченной вероятностью ошибки. Алгоритм Фрейвальдса для проверки умножения матриц.

\item Вероятностный алгоритм для сравнения строк на расстоянии и алгоритм Рабина-Карпа.

\item Вероятностный вариант быстрой сортировки (Randomized QuickSort).

\item Проверка равенства полиномов. Лемма Шварца-Циппеля.

\item Вероятностная проверка простоты числа: алгоритм Соловея-Штрассена.

\item Хеш-таблицы. Универсальные семейства хеш-функций и их применение к хеш-таблицам.

\item Совершенное хеширование.

\item (нет в этом конспекте) Алгоритм Борувки и линейный вероятностный алгоритм для минимального остовного дерева (без алгоритма верификации и поиска тяжелых рёбер).

\item Слабоэкспоненциальный детерминированный алгоритм для 3-SAT.

\item Слабоэкспоненциальный вероятностный алгоритм для 3-SAT (алгоритм Шонинга, использующий случайное блуждание).

\item (нет в этом конспекте) Онлайн-алгоритм для задачи кеширования.

\item (нет в этом конспекте) Алгоритм для онлайн-варианта задачи о покрытии множествами
    
\end{enumerate}
