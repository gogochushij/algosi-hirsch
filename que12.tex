\hypertarget{Фрейвальдс}{\section{Вероятностные алгоритмы с односторонней ограниченной вероятностью ошибки. Алгоритм Фрейвальдса для проверки умножения матриц.}}

\statement{Вероятностные алгоритмы с односторонней ограниченной вероятностью ошибки.}{Нам надо что-то проверить. Придумывает алгоритм, который это проверяет, но может ошибиться с вероятностью p. Повторив его 10 раз, получим вероятность ошибки $p^{10}$, что, вероятно(ha ha), гораздо меньше.}

\statement{Алгоритм Фрейвальдса для проверки умножения матриц.}{Есть три матрицы: $A_{m,n}$, $B_{n,k}$ и $C_{m,k}$, хотим узнать A$\times$B=C или нет.}

\statement{Решение за O($\frac{mnk}{min(m,n,k)}$) с вероятностью ошибки $\le\frac{1}{2}$.}{\\Если все квадратное, то мы проверим за O($n^2$), а умножение матриц - это O($n^{2.8}$)}

Сгенерируем случайный столбец r длины k из нулей и единиц(все равновероятно). Давайте проверять равенство $AB\times r=C\times r$; если умножать так: $A\times (B\times r)$, получится O(nk[$B\times r$]+mn[$A\times Br$]+mk[$C\times r$])=O($\frac{mnk}{min(m,n,k)}$).
\\
\statement{Убедимся в том, что у алгоритма все хорошо}{\\Очев, если $A\times B=C$, алгоритм скажет, что все хорошо.
\\Посчитаем вероятность ошибки.
\\(AB-C)r=0, X=AB-C, посмотрим на ненулевой элемент $x_{kl}$. 
\\ $\sum_{i=1, i\ne l}^{n} x_{ki}r_i+x_{kl}r_l=0,
\\то есть r_l=\frac{\sum}{x_{kl}}$, то есть вероятность ошибки $\le\frac{1}{2}$}. $\blacksquare$