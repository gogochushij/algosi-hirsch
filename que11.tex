\let\bf\bfseries
\let\it\itshape
\section{(11) Алгоритмы Прима и Крускала для задачи о минимальном остовном дереве (\groth)}
\begin{definition}
	Для связного графа $G=\langle V,E\rangle$ {\bf остовным деревом} называется подграф $G'=\langle V,E'\rangle,E'\subseteq E$, который является деревом.
\end{definition}
\begin{problem}
	В связном взвешенном неориентированном графе $G=\langle V,E\rangle$ с весовой функцией $w\colon E\to\mathbb{R}$ найти остовное дерево минимального веса.
\end{problem}
Вспомним, что подмножества ребер графа, в которых нет циклов, являются независимыми в {\it цикловом матроиде}. Наша задача превращается в поиск базы минимального веса, для которого можно использовать {\it жадный алгоритм}\footnote{в курсе комбинаторики был поиск множества максимального веса, но это, по большому счету, одно и то же, потому что можно инвертировать все веса}: начиная с пустого множества последовательно добавляем ребра минимального веса, пока можем.

Алгоритмы Прима и Крускала являются реализациями этого подхода. Остается только научиться быстро находить ребро минимального веса, который можно добавить, чтобы множество осталось независимым.
\begin{definition}
	{\bf\it Разрезом} графа $G=\langle V,E\rangle$ называется пара $(S,T),S,T\subseteq V$, что $V=S\sqcup T$. Ребро называется {\bf\it пересекающим разрез}, если концы ребра находятся в разных множествах разреза. Разрез называется {\bf\it согласованным} со множеством $A$, если никакое ребро из $A$ не пересекает разрез. Ребро называется {\bf легким}, если оно пересекает разрез и имеет минимальный вес среди всех таких ребер, пересекающих разрез.
\end{definition}
\begin{theorem}
	В графе $G=\langle V,E\rangle$ с весовой функцией $w$ $A\subseteq E$~-- независимое подмножество, согласованое с разрезом $(S,T)$, ребро $(u,v)$~-- легкое. Тогда $A\cup\{(u,v)\}$~-- подмножество базы минимального веса, содержащего $A$.
\end{theorem}
\begin{proof}
	Понятно, что множество будет независимым: если это нет так, то еще какое-то ребро пересекает разрез.
	
	Пусть $M$~-- база минимального веса, содержащая $A\cup\{(u,v)\}$, $M'$~-- другая база минимального веса, содержащая $A$. Если она не содержит $(u,v)$, то она содержит какое-то другое ребро $(x,y)$, пересекающее разрез, но тогда $w(M')=w(M)-w((u,v))+w((x,y))\ge w(M)$. Но также $w(M')\le w(M)\Rightarrow w(M')=w(M)$.
\end{proof}
\begin{corollary}
	$G=\langle V,E\rangle,w$~--- неориентированный взвешенный связный граф. $A\subseteq E$~-- независимое множество в его цикловом матроиде. $C=\langle V_C,E_C\rangle$~-- компонента связности леса $G_A=\langle V,A\rangle$. Если $(u,v)$~-- легкое ребро, которое соединяет $C$ с другой компонентой связности, то $A\cup\{u,v\}$~--- подмножество базы минимального веса, содержащего $A$.
\end{corollary}
\begin{proof}
	Разрез $(V_C,V\smallsetminus V_C)$ согласован с $A$ и $(u,v)$ его пересекает.
\end{proof}
\subsection{Алгоритм Крускала}
Этот алгоритм использует структуру данных, которая называется
\subsubsection{Система непересекающихся множеств}


\subsection{Алгоритм Прима}
Это тоже вариант жадного алгоритма. Он похож на алгоритм Дийкстры, в частности, использует {\it очередь с приоритетами} (она была у Охотина в  \href{https://users.math-cs.spbu.ru/~okhotin/teaching/algorithms_2019/okhotin_algorithms_2019_l4.pdf}{лекции 4}, а описание реализации с помощью кучи в \href{https://users.math-cs.spbu.ru/~okhotin/teaching/algorithms_2019/okhotin_algorithms_2019_l5.pdf}{лекции 5}).

Этот алгоритм находит остовное дерево, сторя его из корня $r\in V$.
\begin{algorithm}
	\DontPrintSemicolon
	\SetKwIF{If}{ElseIf}{Else}{if}{:}{elif}{else}{end}
	\SetKwFunction{PrimMST}{Prim-MST}
	\SetKwProg{Fn}{def}{:}{end}
	\Fn{\PrimMST{$G=\langle V,E\rangle,w,r\in V$}}{

	}
\end{algorithm}